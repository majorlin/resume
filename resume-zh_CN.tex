% !TEX TS-program = xelatex
% !TEX encoding = UTF-8 Unicode
% !Mode:: "TeX:UTF-8"

\documentclass{resume}
\usepackage{zh_CN-Adobefonts_external} % Simplified Chinese Support using external fonts (./fonts/zh_CN-Adobe/)
%\usepackage{zh_CN-Adobefonts_internal} % Simplified Chinese Support using system fonts
\usepackage{linespacing_fix} % disable extra space before next section
\usepackage{cite}

\begin{document}
\pagenumbering{gobble} % suppress displaying page number

\name{林明杰}

% {E-mail}{mobilephone}{homepage}
% be careful of _ in emaill address
\contactInfo{(+86) 185-5008-6576}{linmingjie@foxmail.com}{集成电路设计 应用工程师}{}
% {E-mail}{mobilephone}
% keep the last empty braces!
%\contactInfo{xxx@yuanbin.me}{(+86) 131-221-87xxx}{}
 

\section{工作经历}
\datedsubsection{\textbf{恩智浦半导体 | 飞思卡尔半导体} 芯片测试工程师|芯片原型验证工程师}{2014.6-2020.7}
\begin{itemize}
  \item 负责芯片的硅后功能验证,验证内容包含数字IP和模拟IP, 测试环境搭建
  \item 担任\textbf{多个芯片验证工作负责人},负责模块分配,进度管理,结果汇总
  \item 负责FPGA的芯片硅前验证,IP的前期测试,软件开发和ROM开发
  \item 负责芯片硅后验证,模拟IP的参数测试,如ADC,DAC,CMP,Power等模块的参数测试
  \item 参与芯片前端验证,芯片功能验证,协助SDK开发验证,熟悉整个芯片开发设计流程
\end{itemize}

\section{项目经历}
\datedsubsection{\textbf{基于FPGA的ASIC原型系统验证}}{2016.11-2020.7}
\begin{itemize}
  \item 该项目主要是将MCU的ASIC设计\textbf{移植到 FPGA 上实现}, 可以以在流片前期实现对数字IP和系统集成的验证,提供软件和ROM的早期开发平台,缩短ASIC的研发周期。
   \begin{itemize}
    \item 模拟模块替换 
    \item 优化FPGA时钟资源,对ASIC时钟树进行必要对修改
    \item RAM和Flash模型替换; 
    \item FPGA和ASIC仿真环境做相应的修改
   \end{itemize}
  \item FPGA实现之后的板级调试, 定位FPGA原型设计中的问题,结合功能仿真和后仿真实现对FPGA原型设计中对问题定位,修复FPGA中的时序问题。
  \item 开发FPAG配套的升级脚本,提高ROM验证效率
\end{itemize}

\datedsubsection{\textbf{基于Python和LabVIEW的自动化测试系统开发}}{2014.11-2020.7}
\begin{itemize}
  \item 该项目主要是为了提高MCU测试效率而开发的一套\textbf{自动测试系统}, 系统分为自动测试和结果收集展示两大部分,其中自动测试部分可以依据测试代码产生各种测试激励,并控制常见的实验室仪器(如示波器,信号发生器,电源等等)对结果进行测量,并根据测试日志判断测试结果;结果收集部分可以将测试日志保存到数据库,并对测试结果进行分析汇总和网页展示。
   \begin{itemize}
    \item 基于Python和串口的命令交互系统,可以实现测试流程控制,命令延时分发,命令结果汇总
    \item 基于串口命令的仪器控制,可以实现MCU对电源,示波器,万用表,信号发生器的控制
    \item 测试结果汇总,数据库存储,基于浏览器网页的测试结果展示和统计
   \end{itemize}
  \item 支持\textbf{测试过程回溯},可以从网页下载测试过程的关键数据
  \item 支持\textbf{自动测试},可以有效的提高芯片测试效率
\end{itemize}

\datedsubsection{\textbf{芯片模拟参数验证}}{2014.11-2020.7}
\begin{itemize}
  \item 负责实现芯片模拟参数验证,为芯片数据手册提供数据支持,基于LabVIEW和NI的测试仪器对芯片的功耗,ADC的静态参数和动态参数,DAC的参数,ACMP的模拟参数进行测试。对芯片进行高低温参数测试
   \begin{itemize}
    \item 芯片功耗测试包含芯片运行模式和低功耗模式下(高低温)的电压电流功耗测试
    \item ADC/DAC参数测试包含有效位测试,DNL, INL, 转换速率等
    \item ACMP测试包含迟滞阈值,传播延时等参数测试
    \item 芯片的高低温测试主要包含芯片稳定性和芯片功能缺陷复现
   \end{itemize}
\end{itemize}

% \section{\faGraduationCap\ 教育背景}
\section{教育经历}
\datedsubsection{\textbf{中国科学院大学},光学工程,\textit{硕士研究生}}{2011.9 - 2014.7}
\datedsubsection{\textbf{武汉理工大学},通信工程,\textit{工学学士}}{2007.9 - 2011.7}

% \section{\faCogs\ IT 技能}
\section{技术能力}
% increase linespacing [parsep=0.5ex]
\begin{itemize}[parsep=0.2ex]
  \item \textbf{编程语言}: Python, Verilog, C, Shell, LabVIEW, tcl
\end{itemize}

\section{个人总结}
有较好的组织能力,学习上善于掌握各种新技术,学习能力强, 在\textbf{恩智浦半导体}担任芯片测试工程师和芯片原型验证开发工程师,担任多个项目负责人,负责开发基于\textbf{Python}和\textbf{LabVIEW}的自动化测试系统。

%% Reference
%\newpage
%\bibliographystyle{IEEETran}
%\bibliography{mycite}
\end{document}