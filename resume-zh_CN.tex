% !TEX TS-program = xelatex
% !TEX encoding = UTF-8 Unicode
% !Mode:: "TeX:UTF-8"

\documentclass{resume}
\usepackage{zh_CN-Adobefonts_external} % Simplified Chinese Support using external fonts (./fonts/zh_CN-Adobe/)
%\usepackage{zh_CN-Adobefonts_internal} % Simplified Chinese Support using system fonts
\usepackage{linespacing_fix} % disable extra space before next section
\usepackage{cite}

\begin{document}
\pagenumbering{gobble} % suppress displaying page number

\name{林明杰}

% {E-mail}{mobilephone}{homepage}
% be careful of _ in email address
\centerline{\sffamily\large{(+86) 185-5008-6576} \textperiodcentered\ {\sffamily\large linmingjie@foxmail.com}}
\centerline{\sffamily\large{系统软件总监 | MCU/SoC 系统软件架构 | 芯片验证 FPGA 原型 | 10+ 年半导体经验}}
\vspace{0.7ex}
% {E-mail}{mobilephone}
% keep the last empty braces!
%\contactInfo{xxx@yuanbin.me}{(+86) 131-221-87xxx}{}
 

\section{工作经历}
\datedsubsection{\textbf{江苏云途半导体有限公司} 系统软件总监}{2020.7-至今}
\begin{itemize}
  \item 负责公司嵌入式系统软件整体架构设计与研发管理,深度参与 MCU 产品定义、架构设计及功能验证
  \item 建立并持续优化系统软件开发流程,覆盖需求管理、设计评审、代码开发、集成验证及发布全流程
  \item 组建并管理系统软件团队,开展技术指导与评审,推动团队技术能力与工程效率持续提升
  \item 负责芯片 FPGA 验证及硅后(Post-Silicon)验证工作,支撑芯片功能、性能及稳定性验证
  \item 支持芯片及软件功能安全开发流程,配合功能安全需求分析、设计实现与验证活动
  \item 推动软件开发流程规范化与标准化,建立缺陷/问题追踪与闭环管理机制,提升软件质量与项目可控性
\end{itemize}

\datedsubsection{\textbf{恩智浦半导体(NXP) | 飞思卡尔半导体} 芯片测试工程师|芯片原型验证工程师}{2014.6-2020.7}
\begin{itemize}
  \item 负责 MCU/SoC 芯片硅后(Post-Silicon)功能验证,覆盖数字 IP 与模拟 IP,搭建并维护芯片验证与测试环境
  \item 担任多个芯片项目验证负责人,负责验证任务分解、资源协调、进度管理及验证结果汇总与问题跟踪
  \item 参与芯片硅前(Pre-Silicon)验证工作,基于 FPGA 平台开展芯片原型验证及 IP 前期功能测试
  \item 负责芯片硅后模拟 IP 参数测试与特性验证,包括 ADC、DAC、CMP 及电源管理模块等
  \item 参与芯片前端验证与系统级功能验证,协助 SDK 与 ROM 软件开发及验证,熟悉完整芯片研发流程
\end{itemize}

\section{项目经历}

\datedsubsection{\textbf{芯片产品定义和设计开发}}{2020.7-至今}
\begin{itemize}
  \item 结合市场需求,应用需求,技术趋势,竞争产品等多方面因素,制定芯片产品规划和路线图
  \item 参与芯片需求分析,芯片架构设计,芯片模块设计和验证
    \item 对ARM内核有较深入的理解,能够调试和分析 ARM内核相关问题
    \item 熟悉常用外设模块的设计和验证,如UART, SPI, I2C, GPIO, ADC, DAC, PWM等
    \item 熟练使用EDA工具进行芯片设计和验证,如xrun, vcs, verdi, simvision, vivado, synplify等
\end{itemize}

\datedsubsection{\textbf{芯片设计开发验证系统搭建}}{2020.7-2021.12}
\begin{itemize}
  \item 制定芯片设计开发验证流程,基于Python脚本实现芯片验证平台环境搭建和回归测试系统
  \item 开发芯片IP代码管理系统,实现芯片IP代码的版本管理和变更跟踪
  \item 了解芯片开发和设计全流程,协助芯片设计团队进行芯片架构设计和模块设计
\end{itemize}

\datedsubsection{\textbf{芯片设计开发与验证}}{2020.7-至今}
\begin{itemize}
  \item 参与芯片RoadMap制定,芯片需求分析,芯片架构设计,芯片模块设计和验证
  \item 负责芯片FPGA验证及硅后验证工作,确保芯片功能和性能符合设计要求
   \begin{itemize}
    \item 负责项目包含YTM321B1Lx, YTM32B1Mx, YTM32B1Hx等MCU芯片的设计开发和验证工作
    \item 为芯片规格参数提供数据支持,对测试结果进行分析和问题定位。
    \item 对芯片应用、芯片设计、芯片测试等有深入了解,能够从多个角度对芯片开发过程提出改善建议
   \end{itemize}
\end{itemize}

\datedsubsection{\textbf{芯片设计开发与验证}}{2021-至今}
\begin{itemize}
    \item 负责芯片失效的技术分析,了解芯片各种失效模式
    \item 熟悉硬件电路设计和软件设计,能够根据失效模式提出针对性的改善意见
    \item 针对芯片物理分析给出合理的分析计划,并对结果进行分析与确认
\end{itemize}
\datedsubsection{\textbf{基于FPGA的ASIC原型系统验证}}{2016.11-2020.7}
\begin{itemize}
  \item 该项目主要是将MCU的ASIC设计\textbf{移植到 FPGA 上实现},可以在流片前期实现对数字IP和系统集成的验证,提供软件和ROM的早期开发平台,缩短ASIC的研发周期。
   \begin{itemize}
    \item 优化FPGA时钟资源,对ASIC时钟树进行必要的修改
    \item FPGA和ASIC仿真环境做相应的修改,开发流程标准化制定
   \end{itemize}
  \item FPGA实现之后的板级调试,定位FPGA原型设计中的问题,结合功能仿真和后仿真实现对FPGA原型设计中的问题定位,修复FPGA中的时序问题。
  \item 开发FPGA配套的升级脚本,提高ROM验证效率
\end{itemize}

\datedsubsection{\textbf{基于Python和LabVIEW的自动化测试系统开发}}{2014.11-2020.7}
\begin{itemize}
  \item 该项目主要是为了提高MCU测试效率而开发的一套\textbf{自动测试系统}, 系统分为自动测试和结果收集展示两大部分,其中自动测试部分可以依据测试代码产生各种测试激励,并控制常见的实验室仪器(如示波器,信号发生器,电源等等)对结果进行测量,并根据测试日志判断测试结果;结果收集部分可以将测试日志保存到数据库,并对测试结果进行分析汇总和网页展示。
   \begin{itemize}
    \item 基于Python和串口的命令交互系统,可以实现测试流程控制,命令延时分发,命令结果汇总
    \item 基于串口命令的仪器控制,可以实现MCU对电源,示波器,万用表,信号发生器的控制
    \item 测试结果汇总,数据库存储,基于浏览器网页的测试结果展示和统计
   \end{itemize}
  \item 支持\textbf{测试过程回溯},可以从网页下载测试过程的关键数据
  \item 支持\textbf{自动测试},可以有效的提高芯片测试效率
\end{itemize}

\datedsubsection{\textbf{芯片模拟参数验证}}{2014.11-2020.7}
\begin{itemize}
  \item 负责实现芯片模拟参数验证,为芯片数据手册提供数据支持,基于LabVIEW和NI的测试仪器对芯片的功耗,ADC的静态参数和动态参数,DAC的参数,ACMP的模拟参数进行测试。对芯片进行高低温参数测试
   \begin{itemize}
    \item 芯片功耗测试包含芯片运行模式和低功耗模式下(高低温)的电压电流功耗测试
    \item 芯片的高低温测试主要包含芯片稳定性和芯片功能缺陷复现
   \end{itemize}
\end{itemize}


% \section{\faGraduationCap\ 教育背景}
\section{教育经历}
\datedsubsection{\textbf{中国科学院大学},光学工程,\textit{硕士研究生}}{2011.9 - 2014.7}
\datedsubsection{\textbf{武汉理工大学},通信工程,\textit{工学学士}}{2007.9 - 2011.7}

% \section{\faCogs\ IT 技能}
\section{技术能力}
% increase linespacing [parsep=0.5ex]
\begin{itemize}[parsep=0.2ex]
  \item MCU / SoC 系统软件架构设计,熟悉 ARM Cortex-M 内核与调试
  \item 外设与驱动:FlexCAN / LIN / ENET / SENT / SAI / UART / SPI / I2C / GPIO / ADC / PWM 
  \item FPGA 原型验证与芯片硅前 / 硅后验证流程
  \item 自动化测试与工具链:Python / LabVIEW / CI / 回归测试
  \item EDA 工具:Vivado / Synopsys VCS / Verdi / SimVision / Xrun
  \item 软件流程与质量体系:版本管理、缺陷管理、代码评审
  \item AUTOSAR / MCAL 开发与验证经验
  \item 技术团队管理与跨部门协作
\end{itemize}

\section{个人总结}
具备 10 年以上 MCU/SoC 芯片研发与系统软件经验,技术背景覆盖 FPGA 原型验证、芯片硅前/硅后验证、系统软件架构与自动化测试。

曾任职于恩智浦半导体,负责多款 MCU 芯片验证与原型开发;现任江苏云途半导体系统软件总监,主导 MCU 系统软件架构设计与团队建设,深度参与芯片产品定义、设计与验证。

兼具技术深度与管理经验,能够从芯片架构、软件系统、验证流程多个层面推动产品成功交付。
%% Reference
%\newpage
%\bibliographystyle{IEEETran}
%\bibliography{mycite}
\end{document}